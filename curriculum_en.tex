\documentclass{article}
\usepackage[cm]{fullpage}
\usepackage{color}
\usepackage{hyperref}

\hypersetup{breaklinks=true,%
pagecolor=white,%
colorlinks=true,%
linkcolor=cyan,%
urlcolor=MyDarkBlue}

\definecolor{MyDarkBlue}{rgb}{0,0.0,0.45}

%%%%%%%%%%%%%%%%%%%%%%%%%%
% Formatting parameters  %
%%%%%%%%%%%%%%%%%%%%%%%%%%

\newlength{\tabin}
\setlength{\tabin}{1em}
\newlength{\secsep}
\setlength{\secsep}{0.1cm}

\setlength{\parindent}{0in}
\setlength{\parskip}{0in}
\setlength{\itemsep}{0in}
\setlength{\topsep}{0in}
\setlength{\tabcolsep}{0in}

\definecolor{contactgray}{gray}{0.3}
\pagestyle{empty}

%%%%%%%%%%%%%%%%%%%%%%%%%%
%  Template Definitions  %
%%%%%%%%%%%%%%%%%%%%%%%%%%

\newcommand{\lineunder}{\vspace*{-8pt} \\ \hspace*{-6pt} \hrulefill \\ \vspace*{-15pt}}
\newcommand{\name}[1]{\begin{center}\textsc{\Huge#1}\\\end{center}}
\newcommand{\program}[1]{\begin{center}\textsc{#1}\end{center}}
\newcommand{\contact}[1]{\begin{center}\color{contactgray}{\small#1}\end{center}}

\newenvironment{tabbedsection}[1]{
  \begin{list}{}{
      \setlength{\itemsep}{0pt}
      \setlength{\labelsep}{0pt}
      \setlength{\labelwidth}{0pt}
      \setlength{\leftmargin}{\tabin}
      \setlength{\rightmargin}{\tabin}
      \setlength{\listparindent}{0pt}
      \setlength{\parsep}{0pt}
      \setlength{\parskip}{0pt}
      \setlength{\partopsep}{0pt}
      \setlength{\topsep}{#1}
    }
  \item[]
}{\end{list}}

\newenvironment{nospacetabbing}{
    \begin{tabbing}
}{\end{tabbing}\vspace{-1.2em}}

\newenvironment{resume_header}{}{\vspace{0pt}}


\newenvironment{resume_section}[1]{
  \filbreak
  \vspace{2\secsep}
  \textsc{\large#1}
  \lineunder
  \begin{tabbedsection}{\secsep}
}{\end{tabbedsection}}

\newenvironment{resume_subsection}[2][]{
  \textbf{#2} \hfill {\footnotesize #1} \hspace{2em}
  \begin{tabbedsection}{0.5\secsep}
}{\end{tabbedsection}}

\newenvironment{subitems}{
  \renewcommand{\labelitemi}{-}
  \begin{itemize}
      \setlength{\labelsep}{1em}
}{\end{itemize}}

\newenvironment{resume_employer}[4]{
  \vspace{\secsep}
  \textbf{#1} \\ 
  \indent {\small #2} \hfill {\footnotesize#3 (#4)}
  \begin{tabbedsection}{0pt}
  \begin{subitems}
}{\end{subitems}\end{tabbedsection}}


%%%%%%%%%%%%%%%%%%%%%%%%%%
%     Start Document     %
%%%%%%%%%%%%%%%%%%%%%%%%%%

\begin{document}

\begin{resume_header}
	\name{Otavio Sartorelli de Toledo Piza}
	\program{Computer Science Honors \& Data Science Double Major}
	\contact{otaviostpiza@gmail.com \hspace{1cm} \href{https://github.com/OtavioPiza}{github.com/OtavioPiza} \hspace{1cm} +55 11 996298882 \hspace{1cm}Student ID: 0032690213}
\end{resume_header}

\begin{resume_section}{About Me}
  
  \begin{nospacetabbing}
  	\textbf{Technical Skills}  \= Python, Java, Jupyter Notebook, C/C++, Shell, Unix/Linux, Git, \LaTeX \\*
  	\textbf{Frameworks} \> Express, React, MongoDB, PyTorch with Fastai \\*
  	\textbf{Languages} \> Portuguese (native), English (C2), German (B2), Spanish (B2) \\*
  	\textbf{Interests} \> Analysis and Design of Algorithms, Machine Inteligence, Classical Music, Literature, Photography\\*
  \end{nospacetabbing}

\end{resume_section}

\begin{resume_section}{Personal Projects}

	\begin{resume_subsection}{Algorithm Visualizer -- \href{https://www.github.com/OtavioPiza/algorithm-visualizer}{github.com/OtavioPiza/algorithm-visualizer}}
		Pursing my mission of making computer science more accessible, I created a website \textbf{using ReactJS} that allows users to \textbf{visualize how algorithms} work by directly \textbf{interacting with them}, controlling their flow, and \textbf{manipulating the input data}. 
	\end{resume_subsection}
	
	\begin{resume_subsection}{Project Euler 100 -- \href{https://www.github.com/OtavioPiza/project-euler}{github.com/OtavioPiza/project-euler}}
		I created this project to give people insight into the \textbf{design process of efficient algorithms}, iterating through many possible solutions: \textbf{explaining their strengths and weaknesses}. The project consists of a series of Jupyter Notebooks where \textbf{all my logic is explained}, and the \textbf{performance of each solution is demonstrated with plots}. I am currently studying different possibilities to host them online.  
	\end{resume_subsection}

\end{resume_section}

\begin{resume_section}{Work Experience}

	\begin{resume_subsection}[Sao Paulo, SP (2021.06 -- Present)]{BRASA -- Summer Intern}
		I worked on the Summer Journey team (a program where students form groups to solve challenges proposed by leading Brazilian companies in many sectors, such as the financial and retail ones) where I was responsible for the user experience of the companies and students in the program.
	\end{resume_subsection}

\end{resume_section}

\begin{resume_section}{Education}
  
  \begin{resume_subsection}[West Lafayette, IN (2020.05 -- Present)]{Purdue University}
    
    \begin{subitems}
      \item Bachelor of Computer Science Honors
      \item Bachelor of Data Science
      \item Grade Point Average: 3.97 / 4
    \end{subitems}
  
  \end{resume_subsection}
  
  \begin{resume_subsection}[]{Relevant Corsework}
  	
	\begin{subitems}
		\item Data Structures and Algorithms (Purdue CS 251) -- Fall 2021
		\item Web Application Development (Purdue CS 390) -- Fall 2021
		\item Practical Deep Learning for Coders (FASTAI) -- Summer 2021
		\item Fullstack Development (Full Stack open 2021) -- Fall 2020
	\end{subitems}  	
  	
  \end{resume_subsection}
  
\end{resume_section}

\begin{resume_section}{Student Organizations}

\end{resume_section}

\end{document}